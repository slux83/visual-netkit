\chapter{Stato dell'arte}\label{capitolo:arte}
\markboth{Stato dell'arte}{}

In questo capitolo di apertura vogliamo fornire una breve descrizione dei sistemi di \emulazione{} di reti di computers esistenti mettendoli a confrondo per capire le loro potenzialità; in parlicolar modo verrà descritto come questi si comportano davanti ad un utente che vuole creare configurazioni per reti complesse.

Nella seconda parte verranno discusse le qualità della prima release\footnote{Release, letteralmente ``rilascio'', in ambito informatico indica una particolare versione di un software resa disponibile ai suoi utenti, univocamente identificata da altre particolari versioni rese disponibili in precedenza da un particolare numero di versione.} di \visualnetkit{} e, come si vedrà nei successivi capitoli, verranno enfatizzati i punti deboli di questa prima versione che verrà messa a paragone con le pontenzialità offerte dal nuovo e più flessibile ambiente offerto dalla nuova release.

\section{Panoramica sui sistemi di \emulazione{}}
Prima di concentrare i nostri sforzi sul capire cosa rappresenta un ``ambiente di \emulazione{}'' focalizziamo l'attenziene sul significato stesso della parola \emulazione{}. Un software di \emulazione{} o più comunemente chiamato ``emulatore'' è un programma che permette l'esecuzione di software scritto per un ambiente (hardware o software) diverso da quello sul quale l'emulatore stesso viene eseguito.

Un programma scritto e compilato per una determinata piattaforma software (ad esempio \windows) non viene eseguito su un computer con sistema operativo differente come \linux{}. In questi casi si crea sulla macchina ospitante ``Host'' un emulatore che riproduce virtualmente l'ambiente che è stato usato per creare quel programma (nel nostro esempio un ambiente \windows{}). Il sistema virtuale che gira all'interno di quello emulato viene chamato sistema ``Guest''.

I sistemi di \emulazione{} di reti nascono con l'intento di riprodurre il funzionamento delle reti reali e di tutti i servizi che esse offrono agli utenti; configurazioni particolari, protocolli, ecc\ldots

Una rete di calcolatori può essere definita come un sistema informatico costituito da due o più calcolatori che, collegati tra loro tramite un mezzo trasmissivo, possono scambiarsi informazioni di vario genere. Naturalmente nella realtà le cose sono molto più complesse che una semplice definizione. Proprio per questo motivo un sistema di \emulazione{} deve offrire la possibilità di gestire qualsiasi configurazione e comportamento che i calcolatori presenti in rete possono assumere, permettendo così una rappresentazione eterogenea di macchine che offrono diversi servizi e svolgono diverse attività.

\subsection{Classificazione dei sistemi di \emulazione{}}

\subsection{Confronto tra sistemi di \emulazione{} conosciuti}


\section{Ambienti di configurazioni a confronto}

\subsection{Creazione di configurazioni di reti complesse}

\subsubsection{Sistemi a confronto}

\subsubsection{Interfacce grafiche di supporto}

\subsubsection{Flessibilità delle interfaccie grafiche nei sistemi esistenti}

\subsubsection{La prima versione di \visualnetkit{}}


